%SECCIÓN 3. PREGUNTAS INVESTIGACION
\section{Reconocimiento de la información}
Quizas uno de los pasos más importantes  en el proceso de minería de datos se encuentra en el reconocimiento de la información disponible que forma parte del dominio del problema, el problema u objetivo que se busca analizar con dichos información y la caracterización de los diferentes datos que serán objeto del estudio.

 \subsection{Dominio de los datos y conjunto de datos.}
 La información relacionada con los precios de la gasolina en Colombia, es sin duda un tema de interéz común para las instituciones estatales, privadas y ciudadanos en general, esto particularmente por la incidencia que tienen dichos precios en casi cualquier escenario, como el costo de vida de los ciudadanos, precios de operación de las empresas, transporte, etc. Es así como la información relacionada con el comportamiento de los precios de la gasolina en el país puede permitir la identificación de zonas críticas que pueden ser mayormente afectadas por estas fluctuaciones.

Para este análisis se utilizaron los datos  del dataset “Precios de Combustibles” provistos por el portal de datos abiertos Colombia \href{https://datos.gov.co}{https://datos.gov.co}, el cual registra los precios de combustible en diferentes regiones del pais durante el 2016. A continuación se describen los campos que conforman este conjunto de datos: 
 \begin{itemize}
  \item{CodigoDepartamento:} Este campo es el identificador único para cada uno de los departamentos de Colombia
  \item{NombreDepartamento:} Este campo constituye el nombre por el cual son conocidos los diferentes de partamento de Colombia. 
  \item{CodigoMunicipio:} Este campo es el identificador único para cada uno de los municipios de Colombia. En este set de datos, las ciudades tambien se incluyen en la lista de municipios. 
  \item{Municipio:} Nombre con el cual son reconocidos los municipios.	
  \item{Nombrecomercial:} Nombre del establecimiento comercial de expendio de gasolina. 
  \item{Bandera: } Nombre de la marca de expendio de gasolina, por ejemplo texco. 
  \item{Direccion:} Ubicación del establecimiento  de expendio de gasolina deacuerdo a la nomenclatura catastral de la ciudad. 
  \item{Producto:} Nombre del combustible en venta.
  \item{Precio:} valor comercial por galón del combustible en venta.
  \item{Estado:} Describe la disponivilidad del producto con los parámetros activo o inactivo.
  \item{Fecharegistro:} Fecha en la que fue registrado el precio de la gasolina.
  \item{Periodo:} Fecha del primer día del més en que fue realizado el registro del precio de gasolina.
 \end{itemize}
 \subsection{Objetivo SMART}
 Verificar el comportamiento de los precios de la gasolina en Colombia para establecer tendencias y comportamientos en los diferentes departamentos del pais.\\ Esta información podría ser utilizada en la troma de decisiones sobre regulaciones en los precis de la gasolina.
