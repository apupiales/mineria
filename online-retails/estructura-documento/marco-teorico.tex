\section{Marco Teorico}
El análisis de datos representa un importante rol in la historia de la humanidad. Puede implementarse en diferentes ramas del conocimiento y en cualquier tipo de dato. Con el florecimiento de nuevas técnicas de datos y de la expansión y evolución sistemática en mejores motores de almacenamiento se ha generado un avance significativo en las diferentes técnicas y herramientas usadas en minería de datos como resultado natural de ésta  evolución tecnológica. La aplicación de las técnicas de análisis de datos son muy diversas debido a que se puede establecer medidas para ser analizadas en cualquier campo, como ejemplo en datos espaciales, datos de texto, multimedia, networking, WWW, inteligencia de negocios, análisis de ADN, entre otros [1].
La Minería de datos es la tarea de descubrir patrones ocultos de grandes cantidades de datos donde los datos pueden estar almacenados en bases de datos, data warehouses, OLAP (proceso analítico en línea) u otros repositorios de información. También se define como el descubrimiento de conocimiento en bases de datos (KDD por sus sigla en inglés ) [2]. 
La minería de datos involucra una integración de técnicas de múltiples disciplinas tales como tecnologías en bases de datos, estadística, aprendizaje de máquina, redes neuronales, obtención de la información, etc [1].
Según[3]: “La minería de datos es el proceso de descubrimiento de patrones con significado y relaciones que yacen ocultas dentro de grandes bases de datos”. Además [] defina la minería de datos como “el análisis de conjunto de datos observados para encontrar relaciones insospechadas para resumir los datos en nuevas formas que son entendibles y útiles para el propietario de los datos”.
Este proceso consiste básicamente de pasos que son ejecutados antes de llevar a cabo la minería de datos, tales como selección de datos, limpieza de datos, procesamiento y la transformación de datos [4].
La minería de datos es uno de los más importantes campos de investigación gracias a la expansión del desarrollo tecnológico en hardware y software[5]. Los datos son considerados como el activo más importante de cualquier organización, y se puede explotar para predecir decisiones futuras. La minería de datos es una forma de ayuda organizacional para que se haga un total uso de sus datos almacenados[6] .

 \subsection{Proceso de minería de datos estático}
El proceso de minería de datos estático es un paso en el proceso de descubrimiento de conocimiento que consiste en métodos que producen patrones o modelos de los datos. En algunos casos cuando el problema es conocido, la corrección de datos es disponible  existe un intento para encontrar los modelos o herramientas en la cual es usada, algunos problemas pueden ocurrir porque los valores están duplicados, perdidos, erróneos o sobredimensionados y en algunas ocasiones una necesidad para implementar métodos estadísticos también pueden ejecutarse [7].
Kenji [8] definió un modelo de datos semiestructurados y patrones a través de árboles ordenados etiquetados y estudió el problema del descubrimiento de patrones frecuentes usando árboles de decisión con soporte mínimo y superior dentro de una colección de datos semi-estructurados.. Raed  Kersting [9] identificaron algunos conceptos claves y técnicas sobre aprendizaje lógico probabilístico. Y explicaron las diferencias entre los diferentes enfoques y a su vez algunas reflexiones dentro de los desafíos pendientes en el aprendizaje lógico probabilístico. Las técnicas del aprendizaje lógico probabilístico fueron analizadas empezando de una perspectiva lógica (programación lógica inductiva). Además, los principios de aprendizaje estadístico y programación lógica inductiva ( o minería de datos multi-relacional) son empleados para el aprendizaje de parámetros y estructura de la lógica probabilística. Jin y Ag [10] presentaron el diseño y una evaluación de desempeño inicial con un middleware, permitiendo un rápido desarrollo de las aplicaciones de minería de datos en paralelo, el cual puede explotar el paralelismo en plataformas de memoria compartida y memoria distribuida dividiendo instancias  de datos ( o registros de transacciones) entre diferentes nodos. 

 \subsection{Procedimientos de la minería de datos}
Los procedimientos de minería de datos son explicado a continuación que incluyen cinco procesos: Definición del problema de minería de datos, obtención de los datos, detección y corrección de los datos, estimación y construcción del modelo, descripción del modelo [1].
\subsubsection{Definición del problema de minería de datos}
La mayoría de los estudios de modelamiento basados en datos son ejecutados para un dominio particular. Por lo tanto, el conocimiento de un dominio específico y la experiencia son usualmente necesarios con el objetivo de obtener definiciones del problema significativas. 
\subsubsection{Obtención de los datos}
Este proceso concierne a la obtención de datos de diferentes fuentes y locaciones. Los métodos actuales usan fuentes almacenados en bases de datos, data-warehouse, OLAP entre otras.
\subsubsection{Detección y corrección de los datos}
El conjunto de datos en crudo, el cual está inicialmente preparado para la minería de datos son a menudo extensos, sujetos a ruido, pérdida e inconsistencia en la mayoría de ocasiones representan  grandes tamaños de datos.
\subsubsection{Estimación y construcción del modelo de minería de datos}
Este proceso incluye cuatro partes: selección de las tareas de minería, selección de los métodos de minería de datos, selección del algoritmo adecuado y extracción.
\subsubsection{Validación}
La validación del modelo es una condición necesaria pero insuficiente para la credibilidad y aceptabilidad de los resultados de minería. El último objetivo del proceso de minería de datos es proveer un resultado con la suficiencia de credibilidad, aceptación e implementación.
\subsection{Técnicas de minería de datos}
Entre los más importantes métodos se encuentras los Arboles de decisión, redes neuronales, Clustering, reglas de asociación, entre otros [1].

 
