\section{Marco Teorico}
Diferentes ramas de conocimiento permiten caracterizar los comportamientos y tendencias que presentan los datos. Entre ellas se tienen:
 \subsection{Estadísticos}
 Es el estudio de los fenómenos aleatorios para obtener conclusiones acerca del conjunto de datos.
 \subsection{Conjunto de datos}
 Hace parte de los procesos estadísticos, debido a que su colección de registros es usada para describir la población
 \subsection{Minería de datos}
 Es el uso de herramientas estadísticas y de enfoques informáticos para encontrar tendencias en un conjunto de datos o dataset.
 \subsection{Aprendizaje de máquina}
 Es un conjunto de técnicas usadas en minería de datos para encontrar relaciones y predecir el comportamiento de un dataset usando algunos algoritmos de inteligencia artificial.
 \subsection{BIG DATA Analítica}
 Es la combinación entre inteligencia de negocios y técnicas analíticas que son usadas en el descubrimiento de conocimiento de datos (minería de datos) y en análisis estadístico. Algunas de las técnicas relacionadas son: K-Means, Clasificadores Bayesianos, clustering, regresión, el K-más cercano.
 
