%SECCIÓN 3. PREGUNTAS INVESTIGACION
\section{Preguntas de Investigación}
Teniendo claro el escenario del problema, se plantean las siguientes interrogantes que se buscan ser aclaradas a partir de la información suministrada por el conjunto de datos.
 \begin{itemize}
  \item{Preguntas descriptivas.}
  Este tipo de preguntas se utilizan para identificar características del la fuente de datos.
  \begin{itemize}
   \item ¿Cuál fué el departamento que tuvo el mayor precio de gasolina promedio en el 2016?
   \item ¿Cuál fué el departamento que tuvo el menor precio de gasolina promedio en el 2016?
   \item ¿Cuál fue el valor máximo de la gasolina en el país en el 2016?
   \item ¿Cuál fue el valor mínimo de la gasolina en el país en el 2016?
  \end{itemize}
  \item{Preguntas exploratorias.}
   Este tipo de preguntas buscan identificar patrones y relaciones que den soporte a las conclusiones obtenidas sobre una pregunta de investigación.
  \begin{itemize}
   \item ¿Qué bandera maneja los precios más altos de la gasolina en Colombia? 
   \item ¿En qué mes se tuvo el precio más alto de la gasolina?
   \item ¿En qué mes se tuvo el precio más bajo de la gasolina?  
   \item ¿En qué época del año se tuvo el mayor precio promedio?
  \end{itemize}
  \item{Preguntas de inferencia}
   Una pregunta de inferencia parte de una hipótesis que debe ser probada o rechazada a partir de la información analizada.
  \begin{itemize}
   \item ¿Fué Cundinamarca el departamento con el menor precio pormedio de gasolina en julio de 2016?
  \end{itemize}
 \end{itemize}