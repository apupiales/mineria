%SECCIÓN 3. PREGUNTAS INVESTIGACION
\section{Preguntas de investigacion}
El exito de una investigacion radica en la oportuna definicion de lo que se va a realizar, por ello las preguntas de investigacion se clasifican en varios tipos de acorde a la intencion del analisis.
 \begin{itemize}
  \item{Caracter descriptivo.}
  Las preguntas de caracter descriptivo sirven para identificar y conocer las caracteristicas del conjunto de datos.
  \begin{itemize}
   \item ¿Cual es el rango de fechas en la medicion?
   \item ¿Cual es el valor promedio por unidad?
   \item ¿Cual es el producto mas vendido?
   \item ¿Cual es el pais con mayor numero de transacciones?
  \end{itemize}
  \item{Caracter exploratorio.}
   Las preguntas de caracter exploratorio consisten en la busqueda de patrones o relaciones que soporten una pregunta de investigacion
  \begin{itemize}
   \item ¿Cual fue el país que compro la mayor cantidad de productos el mes de enero del año 2012?
   \item ¿Cual fue el mes que mayor valor registro en las transacciones?
   \item ¿Se registraron transacciones por mayor valor que 100000?
   \item ¿Cual es el cliente que menos gasto dinero?
  \end{itemize}
  \item{Caracter inferencial.}
   Las preguntas de caracter inferencial consisten en el planteamiento de una hipotesis que podria ser resuelta con el analisis respectivo de la informacion
  \begin{itemize}
   \item ¿Fue Francia el país que gasto mas dinero?
  \end{itemize}
  \item{Caracter predictivo.}
   Las preguntas de caracter predictivo permiten analizar el comportamiento de la informacion a traves del tiempo, para de esta forma descubrir, proyectar, o realizar hipotesis sobre estados futuros.
  \begin{itemize}
   \item Pendiente
  \end{itemize}
 \end{itemize}