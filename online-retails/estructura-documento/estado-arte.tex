\section{Estado del arte} 
Diferentes enfoques para  predicción de precios han implementado métodos como  árboles de decisión, regresión linear y clustering con resultados importantes.  Yang Lan Qin [11] usaron modelos de regresión lineal para predecir el precio de un filamento de poliéster como apoyo a la cadena de abastecimiento. Otras investigaciones usan técnicas de predicción de precios auto-adaptativos  algoritmos basados en Markov [12]. Para la predicción de costos de inventario se han usado algoritmos de minería de datos basados en redes neuronales y SVM (Support vector machine) [13]. Otros enfoques para la predicción del precio del té en el mercado usando el algoritmo RELIEF [14]. Para la predicción de los cargos a los tiquetes aéreos se han usado algoritmos multi-estratégia (Ripper, Q-Learning y series de tiempo ) de otras investigaciones [15]. También grafos basados en atributos han permitido la detección de fraudes en transferencia de precios  [16]. Con respecto a la predicción de indicadores del precio de gasolina se han usado diferentes técnicas de agrupamiento (clustering) [16] 
